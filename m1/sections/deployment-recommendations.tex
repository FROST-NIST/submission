\section{Deployment Recommendations}\label{section:deployment-recommendations}

% A set of deployment requirements and recommendations, including those
% related to security. This section should also include a list of known and
% proposed applications of the submitted threshold scheme(s).

\subsection{Key Share Storage}

Participants should store their key shares securely, following common secure
practice to store secrets, such as making them readable only by the owner and
storing them encrypted by the operating system keyring or keychain.

Note that it is technically possible for a participant to recover a lost share
with the help of a threshold of other participants. However, this feature is
not covered by this submission.


\subsection{Managing State}

As detailed in Section~\ref{section:comparisons:statemanagement} participants must
manage some state. In each FROST signing session, the participant must keep the
nonces generated in the first round 1, which will be then read again in round 2.
It is recommended that each signing session be assigned some sort of unique
random session ID, so that each participant can map between the session ID and
the nonces for that session. Additionally, nonces should be deleted after they
are used. It is not recommended to persist nonces in long term storage unless
strictly necessarily, to avoid the risk of accidental nonce misuse.

\subsection{Communication Channels}

FROST does not require authenticated nor encrypted channels, as detailed
in Section~\ref{section:system-model}. However, authentication allows identifiable
aborts, which allow pinpointing participants who are not following the protocol
correctly; this can allow the exclusion of that participant for further signing
rounds. Additionally, we expected that encrypted channels might be desired in
many applications due to the private nature of messages being signed.

To authenticate and possibly encrypt communications, we recommend that each
participant should be issued a ``communication keypair'', and each participant
must verify each other's communication public key by some out of band method.
These keypairs can then bootstrap the authenticated and encrypted channels by
using some cryptographic mechanism. We suggest using the Noise protocol (TODO:
ref), specifically the one-way ``X'' pattern which will not add additional
rounds to the FROST protocol. If higher security assurances are needed (such as
stronger forward secrecy), then the ``XK'' pattern might be used, but it adds an
additional round between participants and coordinator to complete the Noise
handshake.

\subsection{Known and Proposed Applications}

It's worth pointing out that most of the proposed applications are not specific
to FROST and could be accomplished by similar threshold signature schemes.

\begin{description}
    \item[Certificate Authorities.] Certificate issuers can use FROST to protect
    issuance keys and make it harder for attackers to compromise a certificate
    authority.
    \item[Legally Binding Eletronic Signatures.] Many jurisdictions allow the
    possibility of using electronic signatures to sign documents. FROST can
    offer an alternative to help secure the critically important private key,
    without needing to resort to self-managed hardware tokens or
    externally-hosted keys inside hardware security modules.
    \item[Cryptocurrency Wallets.] Wallets can use FROST to make them more
    resilient to hacking. FROST is also useful for scenarios where an
    organization need to manage a cryptocurrency wallet with the organization
    funds and multiple persons are required to sign off on spends.
    \item[Cryptocurrency Bridges.] Automated bridges which convert between
    different cryptocurrencies can be built using FROST, where multiple
    distributed nodes manage bridged funds and can issue transfer in response
    to claims for the bridged currency.
    \item[Code Signing.] In many scenarios, such as publishing an app to an
    application store or publishing a software package to a package repository,
    the authenticity of software is guaranteed by a digital signature. Thus, the
    digital signature controls the distribution of software to million of users,
    and ensuring the security of the underlying private key is crucial to
    prevent bad actors from publishing compromised version of those software.
    FROST can help protecting that private key.
    \item[WebAuthn.] The WebAuthn standard specifies mechanisms for
    authenticating users into applications and services. FROST could help
    protect the underlying private key for security-critical credentials.
\end{description}
