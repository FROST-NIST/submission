%!TEX root = ../main.tex

\section{Analytic complexity}\label{section:complexity}

% An analytical estimation of the (i) memory complexity, (ii) computational
% complexity, (iii) communication complexity, and (iv) round complexity of each
% proposed crypto-system. See Section 7.2 about experimental evaluation. As
% applicable, the estimates should include: a breakdown across various phases of
% the protocol; the complexity per party and for the entire system; and the
% functional dependence on configurable parameters, e.g., security strength,
% number of parties and the thresholds.

In the following sections we will refer to three phases:
\begin{itemize}
	\item "Setup" denotes any state that entities need to maintain between
	signing sessions. This is only relevant to memory complexity.
	\item "Round 1" and "round 2" denote the individual signing rounds.
\end{itemize}

We will use the following notation:
\begin{itemize}
	\item $K$: the threshold (minimum number of participants).
	\item $N$: maximum number of participants.
	\item $P$: actual number of participants.
\end{itemize}

\subsection{Round complexity}

FROST requires two rounds to execute a signing session. When using a central
coordinator, each participant receives $1$ inbound message and sends $1$
outbound message per round, while the coordinator correspondingly sends $P$
outbound messages and receives $P$ inbound messages per round.

\subsection{Communication complexity}

In round 1, no information needs to be broadcast from the coordinator to
participants other than the fact that a signing round is starting. The
participants do not even learn which other participants are in the signing
session at this stage. Each participant sends back 2 encoded group elements.

In round 2, each participant receives $2P$ encoded group elements and $P$
encoded scalars, and sends back a single encoded scalar.

Communication complexity at the coordinator scales linearly in the number of
participants: the coordinator downloads $2P$ encoded group elements at the end
of round 1, broadcasts $2P^2$ encoded group elements and $P^2$ encoded scalars
at the beginning of round 2, and downloads $P$ encoded scalars at the end of
round 2.

Note that while the coordinator sends $P$ outbound messages in round 1, the only
information that the protocol requires be conveyed is that a signing round is
starting (as noted above for participants). Empty messages are suitable for this
purpose from an analytic complexity perspective; in a real implementation there
would be some overhead due to message framing.

\begin{table}
	\centering
% BEGIN communication-3-5-3-128
	\begin{tabular}{c c c c}
		\toprule
		Communication & Round & Download & Upload \\ \midrule
		Coordinator & 1 & 192 & 0 \\
		            & 2 & 96 & 864 \\
		            & Total & 288 & 864 \\
		\midrule
		Participant & 1 & 0 & 64 \\
		            & 2 & 288 & 32 \\
		            & Total & 288 & 96 \\
		\bottomrule
	\end{tabular}
% END communication-3-5-3-128
	\caption{Communication complexity for 3-of-5 FROST with 3 participants at the 128-bit security level.}
\end{table}

\begin{table}
	\centering
% BEGIN communication-6-10-8-224
	\begin{tabular}{c c c c}
		\toprule
		Communication & Round & Download & Upload \\ \midrule
		Coordinator & 1 & 912 & 0 \\
		            & 2 & 456 & 10944 \\
		            & Total & 1368 & 10944 \\
		\midrule
		Participant & 1 & 0 & 114 \\
		            & 2 & 1368 & 57 \\
		            & Total & 1368 & 171 \\
		\bottomrule
	\end{tabular}
% END communication-6-10-8-224
	\caption{Communication complexity for 6-of-10 FROST with 8 participants at the 224-bit security level.}
\end{table}

\begin{table}
	\centering
% BEGIN communication-600-1000-700-128
	\begin{tabular}{c c c c}
		\toprule
		Communication & Round & Download & Upload \\ \midrule
		Coordinator & 1 & 44800 & 0 \\
		            & 2 & 22400 & 47040000 \\
		            & Total & 67200 & 47040000 \\
		\midrule
		Participant & 1 & 0 & 64 \\
		            & 2 & 67200 & 32 \\
		            & Total & 67200 & 96 \\
		\bottomrule
	\end{tabular}
% END communication-600-1000-700-128
	\caption{Communication complexity for 600-of-1000 FROST with 700 participants at the 128-bit security level.}
\end{table}

\subsection{Memory complexity}

In the setup phase (before the start of a signing session), the coordinator and
participants need to store $N + 1$ group elements in memory, corresponding to
the group information. Participants additionally need to store 2 scalars: one
representing their identity within the signing group, and the other being their
secret share.

In round 1, the coordinator additionally stores the $2P$ encoded group elements
at the end of round 1; there is no need for these to be decoded (for validity
checking) before broadcasting them in round 2. Participants store an additional
2 group elements and scalars each, along with (TODO explain bytes).

In round 2, the coordinator stores a maximum of $2P + N + 1$ group elements and
$P$ scalars in memory, while participants store a maximum of $2P + N + 3$ group
elements and $P + 4$ scalars. (TODO explain bytes)

TODO: Explain that the tables show the "high water" mark.

\begin{table}
	\centering
% BEGIN memory-3-5-3-128
	\begin{tabular}{c c c c c c}
		\toprule
		Memory & Round & Elements & Scalars & Arb bytes & Cost \\ \midrule
		Coordinator & Setup & 6 & 3 & 0 & 672 \\
		            & 1 & 6 & 3 & 192 & 864 \\
		            & 2 & 12 & 6 & 576 & 1920 \\
		\midrule
		Participant & Setup & 6 & 2 & 0 & 640 \\
		            & 1 & 8 & 4 & 64 & 960 \\
		            & 2 & 12 & 7 & 576 & 1952 \\
		\bottomrule
	\end{tabular}
% END memory-3-5-3-128
	\caption{Memory complexity for 3-of-5 FROST with 3 participants at the 128-bit security level. The cost is in bytes of memory allocated.}
\end{table}

\begin{table}
	\centering
% BEGIN memory-6-10-8-224
	\begin{tabular}{c c c c c c}
		\toprule
		Memory & Round & Elements & Scalars & Arb bytes & Cost \\ \midrule
		Coordinator & Setup & 11 & 8 & 0 & 2337 \\
		            & 1 & 11 & 8 & 912 & 3249 \\
		            & 2 & 27 & 16 & 1789 & 7318 \\
		\midrule
		Participant & Setup & 11 & 2 & 0 & 1995 \\
		            & 1 & 13 & 4 & 114 & 2565 \\
		            & 2 & 27 & 12 & 1789 & 7090 \\
		\bottomrule
	\end{tabular}
% END memory-6-10-8-224
	\caption{Memory complexity for 6-of-10 FROST with 8 participants at the 224-bit security level. The cost is in bytes of memory allocated.}
\end{table}

\begin{table}
	\centering
% BEGIN memory-600-1000-700-128
	\begin{tabular}{c c c c c c}
		\toprule
		Memory & Round & Elements & Scalars & Arb bytes & Cost \\ \midrule
		Coordinator & Setup & 1001 & 700 & 0 & 118496 \\
		            & 1 & 1001 & 700 & 44800 & 163296 \\
		            & 2 & 2401 & 1400 & 67488 & 342784 \\
		\midrule
		Participant & Setup & 1001 & 2 & 0 & 96160 \\
		            & 1 & 1003 & 4 & 64 & 96480 \\
		            & 2 & 2401 & 704 & 67488 & 320512 \\
		\bottomrule
	\end{tabular}
% END memory-600-1000-700-128
	\caption{Memory complexity for 600-of-1000 FROST with 700 participants at the 128-bit security level. The cost is in bytes of memory allocated.}
\end{table}

\subsection{Computational complexity}

The coordinator performs no significant computations in round 1. Participants
perform 2 fixed-base exponentiations, and (for the security levels presented in
this document) process 2 blocks of hash function input.

The significant computations occur during round 2.

Participants perform $2P$ group element multiplications, $P$ exponentiations,
$P + 1$ scalar additions, and $2P + 2$ scalar multiplications. The during
aggregation, the coordinator performs $2P$ group element multiplications, $P$
exponentiations, and $P$ scalar additions.

TODO: Fill out hash function complexity.

TODO: Convey the multi-exp by showing the number of terms in it, and then include
its muls etc. for a specific algorithm in the table (e.g. Pippenger).

\begin{table}
	\centering
% BEGIN computational-3-5-3-128
	\begin{tabular}{c c c c c c c c c c c c c}
		\toprule
		Computation & Round & $\mathsf{Read}_A$ & $\mathsf{Write}_A$ & $A \times A$ & $A^k$  & $B^k$  & $\mathsf{Read}_k$ & $\mathsf{Write}_k$ & $k + k$ & $k \times k$ & $k^n$ & H blocks \\ \midrule
		Coordinator & 1 & 0 & 0 & 0 & 0 & 0 & 0 & 0 & 0 & 0 & 0 & 0 \\
		            & 2 & 6 & 7 & 6 & 3 & 0 & 3 & 15 & 3 & 0 & 0 & 10 \\
		            & Total & 6 & 7 & 6 & 3 & 0 & 3 & 15 & 3 & 0 & 0 & 10 \\
		\midrule
		Participant & 1 & 0 & 2 & 0 & 0 & 2 & 0 & 2 & 0 & 0 & 0 & 2 \\
		            & 2 & 6 & 9 & 6 & 3 & 0 & 3 & 7 & 4 & 8 & 1 & 11 \\
		            & Total & 6 & 11 & 6 & 3 & 2 & 3 & 9 & 4 & 8 & 1 & 13 \\
		\bottomrule
	\end{tabular}
% END computational-3-5-3-128
	\caption{Computational complexity for 3-of-5 FROST with 3 participants at the 128-bit security level.}
\end{table}

\begin{table}
	\centering
% BEGIN computational-6-10-8-224
	\begin{tabular}{c c c c c c c c c c c c c}
		\toprule
		Computation & Round & $\mathsf{Read}_A$ & $\mathsf{Write}_A$ & $A \times A$ & $A^k$  & $B^k$  & $\mathsf{Read}_k$ & $\mathsf{Write}_k$ & $k + k$ & $k \times k$ & $k^n$ & H blocks \\ \midrule
		Coordinator & 1 & 0 & 0 & 0 & 0 & 0 & 0 & 0 & 0 & 0 & 0 & 0 \\
		            & 2 & 16 & 17 & 16 & 8 & 0 & 8 & 80 & 8 & 0 & 0 & 36 \\
		            & Total & 16 & 17 & 16 & 8 & 0 & 8 & 80 & 8 & 0 & 0 & 36 \\
		\midrule
		Participant & 1 & 0 & 2 & 0 & 0 & 2 & 0 & 2 & 0 & 0 & 0 & 2 \\
		            & 2 & 16 & 19 & 16 & 8 & 0 & 8 & 17 & 9 & 18 & 1 & 37 \\
		            & Total & 16 & 21 & 16 & 8 & 2 & 8 & 19 & 9 & 18 & 1 & 39 \\
		\bottomrule
	\end{tabular}
% END computational-6-10-8-224
	\caption{Computational complexity for 6-of-10 FROST with 8 participants at the 224-bit security level.}
\end{table}

\begin{table}
	\centering
% BEGIN computational-600-1000-700-128
	\begin{tabular}{c c c c c c c c c c c c c}
		\toprule
		Computation & Round & $\mathsf{Read}_A$ & $\mathsf{Write}_A$ & $A \times A$ & $A^k$  & $B^k$  & $\mathsf{Read}_k$ & $\mathsf{Write}_k$ & $k + k$ & $k \times k$ & $k^n$ & H blocks \\ \midrule
		Coordinator & 1 & 0 & 0 & 0 & 0 & 0 & 0 & 0 & 0 & 0 & 0 & 0 \\
		            & 2 & 1400 & 1401 & 1400 & 700 & 0 & 700 & 491400 & 700 & 0 & 0 & 1927 \\
		            & Total & 1400 & 1401 & 1400 & 700 & 0 & 700 & 491400 & 700 & 0 & 0 & 1927 \\
		\midrule
		Participant & 1 & 0 & 2 & 0 & 0 & 2 & 0 & 2 & 0 & 0 & 0 & 2 \\
		            & 2 & 1400 & 1403 & 1400 & 700 & 0 & 700 & 1401 & 701 & 1402 & 1 & 1928 \\
		            & Total & 1400 & 1405 & 1400 & 700 & 2 & 700 & 1403 & 701 & 1402 & 1 & 1930 \\
		\bottomrule
	\end{tabular}
% END computational-600-1000-700-128
	\caption{Computational complexity for 600-of-1000 FROST with 700 participants at the 128-bit security level.}
\end{table}
