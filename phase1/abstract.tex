\section{Description of Submission}

\subsection{Primitive to be Thresholdized}

In this work,
we describe FROST,
a two-round threshold signature scheme that thresholdizes the EdDSA signature scheme.
In particular,
we will provide a Cat1 submission for subcategory C1.1 EdDSA signing.

\subsection{Outline of Approach}

FROST (Flexible Round-Optimized Schnorr Threshold Signatures)~\cite{KomloG20,BellareCKMTZ22} addresses the need for efficient threshold EdDSA signing operations while improving upon the state of the art to ensure strong security properties \emph{without} limiting the parallelism of signing operations.
FROST achieves improved efficiency in the optimistic case that no participant misbehaves.
However, in the case where a misbehaving participant contributes malformed values during the protocol,
honest parties can identify and exclude the misbehaving participant, and re-run the protocol.

Here, we describe the (unoptimized) two-round version of the FROST.
However, implementations may wish to perform the first round in a batched setting,
allowing the scheme to be used in a manner where online signing requires only a single round of communication.

\subsubsection{System Model.}
We assume the following when describing FROST:

\begin{itemize}[itemsep=0.5em]
\item \textbf{Idealized Key Generation via Shamir Secret Sharing.} Key generation via Shamir Secret Sharing is an idealized operation.
We assume that each participant is initialized with their respective secret key share,
the public key shares of all other participants,
and the joint public key representing the group.
\item \textbf{Coordinator Role.} We model message passing between participants
via a centralized coordinator.
The coordinator is trusted to not perform denial-of-service attacks by dropping messages,
but otherwise the coordinator is untrusted.
\end{itemize}

\subsubsection{Protocol Approach.}

FROST signing can be performed either in two online rounds,
or in one batched preprocessing round,
and a second purely online round.
Then, a final stage to perform aggregation is required,
at which the joint signature is output.

\paragraph{Round One.}
The first signing round of FROST can be performed either at the time of signing,
or as a batched operation during a pre-processing phase.

Let $\eid$ denote an execution identifier maintained locally by each participant
(which is \emph{not} consistent among all participants).
Each party with identifier $\partyid$ samples two nonces $(r_{\partyid,\eid},  s_{\partyid, \eid}) \randpick \Zp$,
and then derives their corresponding commitments  $R_{\partyid,\eid} \gets g^{r_{\partyid, \eid}}$, $ S_{\partyid,\eid} \gets  g^{s_{\partyid,\eid}}$.
They store $(r_{\partyid,\eid}, s_{\partyid,\eid})$ in their internal state,
and output $(R_{\partyid,\eid}, S_{\partyid,\eid})$.

\paragraph{Round Two.}
All participants in a signing coalition $\coalition$ accept as input a message $\msg$,
and a set of tuples $B = \{ (i, R_{i,\eid}, S_{i,\eid})\}_{i \in \coalition}$,
containing the coalition of parties and their respective commitments.

To begin,
each party derives binding factors $\bindingfactor_i \gets \HashNon(i, B)$ for each party $i \in \coalition$.
Then, each party derives $R \gets \prod_{i \in \coalition} R_{i,\eid} \cdot  S_{i,\eid}^{\bindingfactor_i}$.
Each party derives the challenge as $c \gets \HashSig(R, \pk, m)$,
and then generates their signature share as
$z_\partyid \gets r_{\partyid,\eid} + s_{\partyid,\eid} \cdot \bindingfactor_i + c \cdot \sk_i \cdot \lambda_i$,
where $\lambda_i$ is the Lagrange coefficient.
Each party outputs $z_\partyid$ as their response,
and deletes $(r_{\partyid,\eid}, s_{\partyid,\eid})$.


\paragraph{Aggregation.}
the combiner derives $z = \sum_{i \in \coalition} z_i$,
$R$ is derived as explained above.
The output signature $\sigma = (R,z)$ is a standard Schnorr signature.

\subsubsection{Security Properties.}
FROST achieves a strong notion of unforgeability under the algebraic one-more discrete logarithm (AOMDL) assumption in the random oracle model (ROM) as we discuss in the following.
In general, unforgeability guarantees that an adversary even corrupting the coordinator and up to $t-1$ signers cannot generate a valid signature for $\msg$ that is not considered signed by parties. As noticed by~\cite{BellareCKMTZ22}, there are different levels of conditions to declare $\msg$ is considered signed, which gives different security levels.

\chelsea{It would be helpful to emphasize that basic unforgeability is achieved by $\TSUF{0}$, and that each higher level gives finer-grained guarantees.}

\medskip

\textbf{TS-UF-0.} The weakest condition, which gives the weakest security, referred to as $\TSUF{0}$, considers that $\msg$ was signed as long as at least one honest party generated a signature share for $\msg$. In other words, for a $\TSUF{0}$-secure scheme, the adversary cannot forge a valid signature for $\msg$ if no honest party generated a signature share for $\msg$, but it does not have any security guarantee after the adversary got at least one signature share for $\msg$ even if the adversary does not corrupt any party at all.

\medskip

\textbf{TS-UF-1.} The next level of security, $\TSUF{1}$, provides improved security by strengthening the above condition to requiring that at least $(t - k)$ honest parties generated a signature share for $\msg$,
where $k$ denotes the number of corrupted parties.

\medskip

\textbf{TS-UF-2.} However, $\TSUF{1}$ does not prevent the adversary from combining signature shares from different signing sessions.
More precisely, for a $\TSUF{1}$-secure scheme, the adversary can start several signing sessions for $\msg$ but each with different signing coalitions or different second-round inputs,
and as long as the total number of honest parties that responded in any one of the signing sessions is at least $(t-k)$, the adversary might be able to compute a valid signature for $\msg$.
\chelsea{Wouldn't this result in a ROS attack, and thus break unforgeability altogether?}
Such an malicious behavior is prevented by $\TSUF{2}$, where each signing session is considered independently:
for a signing session for $\msg$, we consider $\msg$ was signed in the session (or the session is valid) only if at least $(t - k)$ honest parties responded in the session, and $\msg$ is considered signed as long as it was signed in one of the signing sessions.

\medskip

\textbf{TS-UF-3.} Further,~\cite{BellareCKMTZ22} shows that FROST achieves the next level of security, $\TSUF{3}$, where the condition that a session is valid is further strengthened.
Namely, for a signing session with coalition $\coalition$ and second-round  input $\{ (i, R_{i,\eid}, S_{i,\eid})\}_{i \in \coalition}$,
in additional to the above condition, it is also required that each
$(R_{i,\eid}, S_{i,\eid})$ that is input into Round 2 is in fact a correct commitment output by honest signer $i$ in Round 1 (i.e., $(R_{i,\eid}, S_{i,\eid})$ is the first-round commitment output by signer $i$)\footnote{Since the coordinator was corrupted,  the coordinator can send arbitrary commitments $(R_{i\eid}, S_{i,\eid})$ in the second round.}.
\chelsea{edited this for clarity, please review to make sure it is correct.}

\medskip

\textbf{TS-UF-4.}  \chelsea{TODO- this can be achieved by an authenticated network channel.}

\medskip
\textbf{TS-SUF-3.} Moreover,~\cite{BellareCKMTZ22} shows that FROST is \emph{strongly} unforgeable, referred to as $\TSSUF{3}$, which, analogous to the strong unforgeability of signature schemes, guarantees that an adversary cannot forge a signature $\sigma$ that is not considered issued for a message $\msg$ even if $\msg$ was signed in some signing sessions. Here, for a signing session, given its coalition and second-round input, it is guaranteed that there exists a unique signature $\sigma$ that can be issued, and $\sigma$ is considered issued only if the session succeeded following the condition from $\TSUF{3}$.
\chelsea{What does it mean for the session to succeed? That a valid signature was output?}



%FROST is statically secure assuming fewer than $t$ parties are corrupted,
%and the algebraic one-more discrete logarithm (AOMDL) assumption holds in the random oracle model (ROM).

%TODO: add a summary here of different TS-UF levels.
