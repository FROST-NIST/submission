\section{Notation}

We give a summary of our notation and where it differs from the notation in the NIST call in Table~\ref{table:notation-comp}.

\begin{table}[htbp]
	\centering
	\begin{tabular}{c | c | c }
		\toprule
    NIST Notation & Our Notation & Meaning \\
    f & \thresh  & Corruption threshold \\
    f-1 & \tcor & Number of supported corruptions \\
		\bottomrule
	\end{tabular}
	\caption{\label{table:comparisons}
		Table of comparisons with our notation versus the notation in the NIST
    call.
	 }
\end{table}


\subsection{General Notation}

We let $\secp \in \N$ denote the security parameter, and let
$\negl$ denote a negligible function.
For a non-empty set $S$, let $x\randpick S$ denote sampling
a uniform element of $S$ and assigning it to~$x$.
We use $\set{n}$ to represent the $\{1,\ldots,n \}$ and $[i..j]$ to represent $\{i,\ldots,j \}$.
We represent vectors as $\vec{a} = ( a_1, \ldots, a_n )$.

PPT stands for ``probabilistic polynomial time.''  Algorithms are randomized unless explicitly noted otherwise.
We let $y \gets A(x;\rho)$ denote running algorithm $A$ on
input $x$ and randomness $\rho$ and assigning its output to $y$.
Let $y \randpick A(x)$
denote $y \gets A(x;\rho)$ for a uniform~$\rho$.
The set of values that have non-zero probability of being output by $A$ on input $x$ is denoted by $[A(x)]$.
We let $\GroupGen$ be a PPT algorithm that takes as input $\usecp$ and outputs a description $(\Gr, p, g)$ of a group $\Gr$ of order a prime $p>2^\secp$, and a generator $g$ of~$\Gr$.

\medskip\noindent{\bf Polynomial interpolation.}
A polynomial $f(x) = a_0 + a_1 x + a_2 x^2 + \cdots + a_{\tcor} x^{\tcor}$
of degree $\tcor$ over a field $\F$ can be interpolated from its value on $\thresh$ points.
For  distinct elements~$S = \{x_1, \ldots, x_{\thresh}\}$ in $\F$,
define the Lagrange polynomial
\begin{equation}\label{eqn:lagrange}
L_i^{(S)}(x) = \prod_{j \neq i } \frac{x-x_j}{x_i - x_j}.
\end{equation}
Given $(x_i, y_i)_{i \in [\thresh]}$, we can implicitly evaluate the corresponding polynomial $f$ at any point~$x$ as
\[ f(x) = \sum_{k \in [\thresh]} f(x_k) \cdot L^{(S)}_k(x). \]


\subsection{Building Blocks}

We next give notation for building blocks that we refer to in the remainder of
this document.

\begin{definition}[Schnorr signatures~\cite{Schnorr91}] \label{defn:schnorr}
The Schnorr signature scheme consists of PPT algorithms $(\Setup, \KeyGen, \Sign, \Verify)$ defined as follows:

  \begin{itemize}[itemsep=1mm]

    \item $\Setup(1^\secp) \rightarrow \pp$:
    On input $1^\secp$, run $(\Gr, p, g) \gets \GroupGen(\usecp)$ and select a hash function $\Hash: \{0,1\}^* \rightarrow \Zp$. Output public parameters $\pp \gets ((\Gr, p, g), \Hash)$ (which are given implicitly as input to all other algorithms).

  \item $\KeyGen() \rightarrow (\pk, \sk)$:
  Sample a secret key $\sk \randpick \Zp$ and compute the public key as $\pk \gets g^{\sk}$.
    Output key pair $(\pk, \sk)$.

  \item $\Sign(\sk, m) \rightarrow \sigma$:
    On input a secret key $\sk$ and a message $m$, sample a nonce $r \randpick \Zp$. Then, compute a
     commitment $R \gets g^r$,  challenge $c \gets \Hash(\pk, m, R)$, and
    response $z \gets r + c \cdot \sk$. Output signature $\sigma \gets (R,z)$.
%    \jnote{I find $\gets$ confusing for deterministic assignment\ldots} \liz{How about $:=$?} \jnote{yes. And then randomized assignment can lose the dollar sign}
%      \chelsea{my vote is to keep the dollar sign, please}

  \item $\Verify(\pk, m, \sigma) \rightarrow 0/1$:
  On input a public key $\pk$,
    a message $m$, and a purported signature $\sigma = (R, z)$, compute $c
    \leftarrow \Hash(\pk, m, R)$ and output $1$ (accept) if $R \cdot \pk^c = g^z$; else, output $0$ (reject).

\end{itemize}
\end{definition}

\begin{definition}[Shamir secret sharing~\cite{Shamir79}] \label{defn:shamir}
The $(\thresh, n)$-Shamir secret sharing scheme consists of algorithms $(\IssueShares, \allowbreak \Recover)$, defined as follows:

\begin{itemize}[itemsep=1mm]
    \item $\IssueShares(\sk, n, \thresh) \rightarrow \{ (1, \sk_1), \ldots, (n, \sk_n)\}$:
      On input a secret $\sk$, number of participants $n$, and threshold $\thresh$, first, define a polynomial $f(Z) = \sk + a_1 + a_2 Z^2 + \cdots + a_{\tcor} Z^{\tcor}$ by sampling $a_1, \ldots, a_{\tcor} \randpick \Zp$. Then, set each participant's share $\sk_i, i \in \set{n}$, to be the evaluation of $f(i)$:
    \begin{equation*} \label{eq:issueshares}
        \sk_i \gets \sk + \sum_{j \in \set{\tcor}} a_j i^j.
    \end{equation*}
    Output $\{(i, \sk_i) \}_{i \in \set{n}}$.

  \item $ \Recover(\thresh, \{ (i, \sk_i)\}_{ i \in \S}) \rightarrow \bot / \sk$:
    On input threshold $\thresh$ and a set of shares $\{(i, \sk_i) \}_{i \in \S}$,
    output $\bot$ if $\S \not\subseteq \set{n}$ or if $\lvert \S \rvert < \thresh$. Otherwise, recover $\sk$ as follows:
    \begin{equation*} \label{eq:recovershares}
      \sk \gets \sum_{i \in \S} \lambda_i \sk_i
    \end{equation*}
    where the Lagrange coefficient for the set
$\S$ is defined as
\[ \lambda_i = \prod_{j \in \S, j\neq i} \frac{j}{j - i} .\]
%(This is the evaluation of $L_i(0)$ in Equation~\ref{eqn:lagrange}.)
\end{itemize}
\end{definition}
