\section{Description of Submission}

\subsection{Primitive to be Thresholdized}

In this work,
we describe FROST,
a two-round threshold signature scheme that thresholdizes the EdDSA signature scheme.
In particular,
we will provide a Cat1 submission for subcategory C1.1 EdDSA signing.

\subsection{Outline of Approach}

FROST (Flexible Round-Optimized Schnorr Threshold Signatures)~\cite{KomloG20,BellareCKMTZ22} addresses the need for efficient threshold EdDSA signing operations while improving upon the state of the art to ensure strong security properties \emph{without} limiting the parallelism of signing operations.
FROST achieves improved efficiency in the optimistic case that no participant misbehaves.
However, in the case where a misbehaving participant contributes malformed values during the protocol,
honest parties can identify and exclude the misbehaving participant, and re-run the protocol.

Here, we describe the (unoptimized) two-round version of the FROST.
However, implementations may wish to perform the first round in a batched setting,
allowing the scheme to be used in a manner where online signing requires only a single round of communication.

\subsubsection{System Model.}
We assume the following when describing FROST:

\begin{itemize}[itemsep=0.5em]
\item \textbf{Idealized Key Generation via Shamir Secret Sharing.} Key generation via Shamir Secret Sharing is an idealized operation.
We assume that each participant is initialized with their respective secret key share,
the public key shares of all other participants,
and the joint public key representing the group.
\item \textbf{Coordinator Role.} We model message passing between participants
via a centralized coordinator.
The coordinator is trusted to not perform denial-of-service attacks by dropping messages,
but otherwise the coordinator is untrusted.
\end{itemize}

\subsubsection{Protocol Approach.}

FROST signing can be performed either in two online rounds,
or in one batched preprocessing round,
and a second purely online round.
Then, a final stage to perform aggregation is required,
at which the joint signature is output.

\paragraph{Round One.}
The first signing round of FROST can be performed either at the time of signing,
or as a batched operation during a pre-processing phase.

Each party with identifier $\partyid$ samples two nonces $(r_\partyid, s_\partyid) \randpick \Zp$,
and then derives their corresponding commitments $R_\partyid \gets g^{r_\partyid}$, $ S_\partyid \gets g^{s_\partyid}$.
They store $(r_\partyid, s_\partyid)$ in their internal state,
and output $(R_\partyid, S_\partyid)$.

\paragraph{Round Two.}
All participants in a signing coalition $\coalition$ accept as input a message $\msg$,
and a set of tuples $B = \{ (i, R_i, S_i)\}_{i \in \coalition}$,
containing the coalition of parties and their respective commitments.

To begin,
each party derives binding factors $\bindingfactor_i \gets \HashNon(i, B)$ for each party $i \in \coalition$.
Then, each party derives $R \gets \prod_{i \in \coalition} R_i \cdot S_i^{\bindingfactor_i}$.
Each party derives the challenge as $c \gets \HashSig(\pk, R, c)$,
and then generates their signature share as
$z_\partyid \gets r_\partyid + s_\partyid \cdot \bindingfactor_i + c \cdot \sk_i \cdot \lambda_i$,
where $\lambda_i$ is the Lagrange coefficient.
Each party outputs $z_\partyid$ as their response,
and deletes $(r_\partyid, s_\partyid)$.


\paragraph{Aggregation.}
the combiner derives $z = \sum_{i \in \coalition} z_i$,
$R$ is derived as explained above.
The output signature $\sigma = (R,z)$ is a standard Schnorr signature.

\subsubsection{Security Properties.}
FROST achieves a strong notion of unforgeability under the algebraic one-more discrete logarithm (AOMDL) assumption in the random oracle model (ROM) as we discuss in the following.
In general, unforgeability guarantees that an adversary even corrupting the coordinator and up to $t-1$ signers cannot generate a valid signature $\sigma$ for $\msg$ that is not supposed to be issued. As noticed by~\cite{BellareCKMTZ22}, there are different levels of conditions to declare a signature for $\msg$ is issued, which gives different levels of security (ordered from weak to strong):
\begin{itemize}
\item $\TSUF{0}$: At least one honest party generated a signature share for $\msg$.
\item $\TSUF{1}$: At least $(t - |\corruptedSigners|)$ honest parties generated a signature share for $\msg$, where $\corruptedSigners$ denotes the set of corrupted signers.
\item $\TSUF{2}$: There exists a second-round signing session with coalition $\coalition$, $\msg$ and some input $\{ (i, R_i, S_i)\}_{i \in \coalition})$ in which at least $(t - |\corruptedSigners|)$ honest parties from $\coalition$ responded.
\end{itemize}
It is shown by~\cite{BellareCKMTZ22} that FROST achieves $\TSUF{2}$.

Moreover,~\cite{BellareCKMTZ22} shows that FROST is \emph{strong} unforgeable, referred to as $\TSSUF{2}$, which additionally guarantees that for each second-round signing session that satisfies the condition of $\TSUF{2}$ mentioned above at most one signature is supposed to be issued and there exists an efficient algorithm that returns whether a signature is supposed to be issued by a signing session given its second-round coalition $\coalition$ and input $(\msg, \{ (i, R_i, S_i)\}_{i \in \coalition})$.


%FROST is statically secure assuming fewer than $t$ parties are corrupted,
%and the algebraic one-more discrete logarithm (AOMDL) assumption holds in the random oracle model (ROM).

%TODO: add a summary here of different TS-UF levels.
