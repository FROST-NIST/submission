\section{Description of Submission}

\subsection{Primitive to be Thresholdized}

In this work,
we describe FROST,
a two-round threshold signature scheme that thresholdizes the EdDSA signature scheme.
In particular,
we will provide a Cat1 submission for subcategory C1.1 EdDSA signing.

\subsection{Outline of Approach}

FROST (Flexible Round-Optimized Schnorr Threshold Signatures)~\cite{KomloG20,BellareCKMTZ22} addresses the need for efficient threshold EdDSA signing operations while improving upon the state of the art to ensure strong security properties \emph{without} limiting the parallelism of signing operations.
FROST achieves improved efficiency in the optimistic case that no participant misbehaves.
However, in the case where a misbehaving participant contributes malformed values during the protocol,
honest parties can identify and exclude the misbehaving participant, and re-run the protocol.

Here, we describe the (unoptimized) two-round version of the FROST.
However, implementations may wish to perform the first round in a batched setting,
allowing the scheme to be used in a manner where online signing requires only a single round of communication.

\subsubsection{System Model.}
We assume the following when describing FROST:

\begin{itemize}[itemsep=0.5em]
\item \textbf{Idealized Key Generation via Shamir Secret Sharing.} Key generation via Shamir Secret Sharing is an idealized operation.
We assume that each participant is initialized with their respective secret key share,
the public key shares of all other participants,
and the joint public key representing the group.
\item \textbf{Coordinator Role.} We model message passing between participants
via a centralized coordinator.
The coordinator is trusted to not perform denial-of-service attacks by dropping messages,
but otherwise the coordinator is untrusted.
\end{itemize}

\subsubsection{Protocol Approach.}

FROST signing can be performed either in two online rounds,
or in one batched preprocessing round,
and a second purely online round.
Then, a final stage to perform aggregation is required,
at which the joint signature is output.

Note that while FROST assumes that signing nonces are used at most once,
it does \emph{not} assume that participants maintain consistent session identifiers.
To reflect this requirement,
we introduce the notation of $\eid$, to denote an execution identifier maintained locally by each participant.
$\eid$ is \emph{not} consistent among all participants.

\paragraph{Round One.}
The first signing round of FROST can be performed either at the time of signing,
or as a batched operation during a pre-processing phase.

Each party with identifier $\partyid$ samples two nonces $(r_{\partyid,\eid},  s_{\partyid, \eid}) \randpick \Zp$,
and then derives their corresponding commitments  $R_{\partyid,\eid} \gets g^{r_{\partyid, \eid}}$, $ S_{\partyid,\eid} \gets  g^{s_{\partyid,\eid}}$.
They store $(r_{\partyid,\eid}, s_{\partyid,\eid})$ in their internal state,
and output $(R_{\partyid,\eid}, S_{\partyid,\eid})$.

\paragraph{Round Two.}
All participants in a signing coalition $\coalition \subseteq \set{n}, \lvert \coalition \rvert \geq t$ accept as input a message $\msg$ and a combination of commitments $\setCommit := \{ (i, R_{i,\eid}, S_{i,\eid})\}_{i \in \coalition}$ from parties in the coalition.

To begin,
each party derives binding factors $\bindingfactor_i \gets \HashNon(i, \setCommit)$ for each party $i \in \coalition$.
Then, each party derives the group commitment $R \gets \prod_{i \in \coalition} R_{i,\eid} \cdot  S_{i,\eid}^{\bindingfactor_i}$.
Each party derives the challenge as $c \gets \HashSig(R, \pk, m)$,
and then generates their signature share as
$z_\partyid \gets r_{\partyid,\eid} + s_{\partyid,\eid} \cdot \bindingfactor_i + c \cdot \sk_i \cdot \lambda_i$,
where $\lambda_i$ is the Lagrange coefficient.
Each party outputs $z_\partyid$ as their response,
and deletes $(r_{\partyid,\eid}, s_{\partyid,\eid})$.


\paragraph{Combine.}
The combiner derives the joint response $z = \sum_{i \in \coalition} z_i$.
The group commitment $R$ is derived as explained above.
The output signature $\sigma = (R,z)$ is a standard Schnorr signature.

\subsubsection{Security Properties.}
FROST achieves a strong notion of unforgeability under the algebraic one-more discrete logarithm (AOMDL) assumption in the random oracle model (ROM) as we discuss in the following.
In general, unforgeability guarantees that an adversary even corrupting the coordinator and up to $t-1$ signers cannot generate a valid signature for $\msg$ that is not considered signed by parties. As noticed by~\cite{BellareCKMTZ22}, there are different levels of conditions to declare $\msg$ is considered signed, which gives different security levels, and stronger conditions give stronger security guarantees.

%\chelsea{It would be helpful to emphasize that basic unforgeability is achieved by $\TSUF{0}$, and that each higher level gives finer-grained guarantees.}

\medskip

\textbf{TS-UF-0.} The weakest condition, which gives weakest unforgeability, referred to as $\TSUF{0}$, considers that $\msg$ was signed as long as at least one honest party generated a signature share for $\msg$. In other words, for a $\TSUF{0}$-secure scheme, the adversary cannot forge a valid signature for $\msg$ if no honest party generated a signature share for $\msg$, but it does not give any security guarantee after the adversary got at least one signature share for $\msg$ even if the adversary corrupted less than $(t-1)$ parties.

\medskip

\textbf{TS-UF-1.} The next level of security, $\TSUF{1}$, provides improved security by strengthening the above condition to requiring that at least $(t - k)$ honest parties generated signature shares for $\msg$,
where $k$ denotes the number of corrupted parties.

However, $\TSUF{1}$ does not guarantee that when the honest parties generated the signature shares for $\msg$, they all had the same view, i.e., receiving the same second-round input.
%igning coalition and the same combinations of commitments.
  %forwarding inconsistent information to different honest parties in the second (online) round.
%combining signature shares from different signing sessions.
More precisely, for a $\TSUF{1}$-secure scheme, the corrupted coordinator can send different combinations of commitments to different honest parties for signing $\msg$ in the (online) second round,
and as long as the total number of honest parties responded is at least $(t-k)$, the adversary might be able to compute a valid signature for $\msg$.

%\chelsea{Wouldn't this result in a ROS attack, and thus break unforgeability altogether?}

\medskip

\textbf{TS-UF-2.}
Such an malicious behavior is prevented by $\TSUF{2}$, where
we consider $\msg$ was signed only if at least $(t - k)$ honest parties generated signature shares for $\msg$ and they received the same commitment combination when generating the shares.

\medskip

\textbf{TS-UF-3.} Further, Bellare et. al.~\cite{BellareCKMTZ22} showed that FROST achieves the next level of security, $\TSUF{3}$, where the above condition is further strengthened:
we declare $\msg$ was signed only if there exists a coalition $\coalition$ and $\setCommit=\{ (i, R_{i}, S_{i})\}_{i \in \coalition}$ such that not only $(t - k)$ honest parties but also all honest parties $i\in \coalition$ with \emph{correct} $(R_{i}, S_{i})$ generated signature shares for the same secound-round input $(\msg, \setCommit)$, where we say $(R_{i}, S_{i})$ is correct if and only if it was output by party $i$ in Round 1.\footnote{Since the coordinator was corrupted, the commitment $(R_{i}, S_{i})$ might not be one of the commitments output by honest party $i$ in Round 1.}

%Namely, for a signing session with coalition $\coalition$ and second-round input $\{ (i, R_{i,\eid}, S_{i,\eid})\}_{i \in \coalition}$,
%in additional to the above condition, it is also required that each
%$(R_{i,\eid}, S_{i,\eid})$ that is input into Round 2 is in fact a correct commitment output by honest signer $i$ in Round 1 (i.e., $(R_{i,\eid}, S_{i,\eid})$ is the first-round commitment output by signer $i$).
%\chelsea{edited this for clarity, please review to make sure it is correct.}

\medskip

\textbf{TS-UF-4.} Bellare et. al.~\cite{BellareCKMTZ22} also showed that if we assume authenticated network channels which guarantee that the corrupted coordinator cannot forward incorrect commitments to honest parties in the online round, FROST achieves a yet stronger notion of unforgeability, $\TSUF{4}$, where we strengthen the $\TSUF{3}$ condition by requiring that there exists a coalition $\coalition$ and $\setCommit=\{ (i, R_{i}, S_{i})\}_{i \in \coalition}$ such that all honest parties in $\coalition$ generated shares for $(\msg, \setCommit)$.
It is a stronger condition since the size of $\coalition$ is at least $t$ and thus the number of honest parties in $\coalition$ is at least $(t - k)$.

\medskip
\textbf{Strong unforgeability.} Moreover, Bellare et. al.~\cite{BellareCKMTZ22} showed that FROST is \emph{strongly} unforgeable, referred to as $\TSSUF{3}$ (or $\TSSUF{4}$ assuming authenticated channels), which, analogous to the strong unforgeability of signature schemes, guarantees that an adversary cannot forge a message-signature pair $(\msg,\sigma)$ that is not considered issued.
Also, it is guaranteed that there is at most one signature $\sigma$ that can be issued for each second-round input $(\msg, \setCommit=\{ (i, R_{i}, S_{i})\}_{i \in \coalition})$, and $(\msg, \sigma)$ is considered issued only if there are \emph{sufficiently many} honest parties generated signature shares for $(\msg, \setCommit)$.
In particular, for $\TSSUF{3}$, ``sufficiently many honest parties'' includes all honest parties $i\in \coalition$ with \emph{correct} $(R_{i}, S_{i})$, and the total number of honest parties must be at least $(t - k)$; for $\TSSUF{4}$, ``sufficiently many honest parties'' refers to all honest parties in $\coalition$.



%FROST is statically secure assuming fewer than $t$ parties are corrupted,
%and the algebraic one-more discrete logarithm (AOMDL) assumption holds in the random oracle model (ROM).

%TODO: add a summary here of different TS-UF levels.
