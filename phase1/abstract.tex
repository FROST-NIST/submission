\section{Description of Submission}

\subsection{Primitive to be Thresholdized}

In this work,
we describe FROST,
a two-round threshold signature scheme that thresholdizes the EdDSA signature scheme.
In particular,
we will provide a Cat1 submission for subcategory C1.1 EdDSA signing.

\subsection{Outline of Approach}

FROST (Flexible Round-Optimized Schnorr Threshold Signatures)~\cite{KomloG20,BellareCKMTZ22} addresses the need for efficient threshold EdDSA signing operations while improving upon the state of the art to ensure strong security properties \emph{without} limiting the parallelism of signing operations.
FROST achieves improved efficiency in the optimistic case that no participant misbehaves.
However, in the case where a misbehaving participant contributes malformed values during the protocol,
honest parties can identify and exclude the misbehaving participant, and re-run the protocol.

Here, we describe the (unoptimized) two-round version of the FROST.
However, implementations may wish to perform the first round in a batched setting,
allowing the scheme to be used in a manner where online signing requires only a single round of communication.

\subsubsection{System Model.}
We assume the following when describing FROST:

\begin{itemize}[itemsep=0.5em]
\item \textbf{Idealized Key Generation.} Key generation is an idealized operation.
We assume that each participant is initialized with their respective secret key share,
the public key shares of all other participants,
and the joint public key representing the group.
\item \textbf{Coordinator Role.} We model message passing between participants
via a centralized coordinator.
The coordinator is trusted to not perform denial-of-service attacks by dropping messages,
but otherwise the coordinator is untrusted.
\end{itemize}

\subsubsection{Protocol Approach.}

FROST signing can be performed either in two online rounds,
or in one batched preprocessing round,
and a second purely online round.
Then, a final stage to perform aggregation is required,
at which the joint signature is output.

\paragraph{Round One.}

\paragraph{Round Two.}

\paragraph{Aggregation and Signature Publication.}

\subsubsection{Security Properties.}




