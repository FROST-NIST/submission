%!TEX root = ../main.tex

\section{Analytic complexity}\label{section:complexity}

% An analytical estimation of the (i) memory complexity, (ii) computational
% complexity, (iii) communication complexity, and (iv) round complexity of each
% proposed crypto-system. See Section 7.2 about experimental evaluation. As
% applicable, the estimates should include: a breakdown across various phases of
% the protocol; the complexity per party and for the entire system; and the
% functional dependence on configurable parameters, e.g., security strength,
% number of parties and the thresholds.

\subsection{Round complexity}

The round complexity of FROST is straightforward: FROST requires two rounds to
execute a signing session.

In the following sections we will refer to three rounds:
\begin{itemize}
	\item "Round 0" denotes any state that entities need to maintain between signing
  sessions. This is only relevant to memory complexity. TODO: Call this Setup.
  \item "Round 1" and "round 2" denote the actual signing rounds.
\end{itemize}

We will use the following notation:
\begin{itemize}
	\item $K$: the threshold (minimum number of participants).
	\item $N$: maximum number of participants.
	\item $P$: actual number of participants.
\end{itemize}

\subsection{Communication complexity}

In round 1, no information needs to be broadcast from the coordinator to
participants other than the fact that a signing round is starting. The
participants do not even learn which other participants are in the signing
session at this stage. Each participant sends back 2 encoded group elements.

In round 2, each participant receives $2P$ encoded group elements and $P$
encoded scalars, and sends back a single encoded scalar.

Communication complexity at the coordinator scales linearly in the number of
participants: the coordinator downloads $2P$ encoded group elements at the end
of round 1, broadcasts $2P^2$ encoded group elements and $P^2$ encoded scalars
at the beginning of round 2, and downloads $P$ encoded scalars at the end of
round 2.

\begin{table}
	\centering
% BEGIN communication-3-5-3-128
% END communication-3-5-3-128
	\caption{Communication complexity for 3-of-5 FROST with 3 participants at the 128-bit security level.}
\end{table}

\begin{table}
	\centering
% BEGIN communication-6-10-8-224
% END communication-6-10-8-224
	\caption{Communication complexity for 6-of-10 FROST with 8 participants at the 224-bit security level.}
\end{table}

\begin{table}
	\centering
% BEGIN communication-600-1000-700-128
% END communication-600-1000-700-128
	\caption{Communication complexity for 600-of-1000 FROST with 700 participants at the 128-bit security level.}
\end{table}

\subsection{Memory complexity}

In "round 0" (before the start of a signing session), the coordinator and
participants need to store $N + 1$ group elements in memory, corresponding to
the group information. Participants additionally need to store 2 scalars: one
representing their identity within the signing group, and the other being their
secret share.

In round 1, the coordinator additionally stores the $2P$ encoded group elements
at the end of round 1; there is no need for these to be decoded (for validity
checking) before broadcasting them in round 2. Participants store an additional
2 group elements and scalars each, along with (TODO explain bytes).

In round 2, the coordinator stores a maximum of $2P + N + 1$ group elements and
$P$ scalars in memory, while participants store a maximum of $2P + N + 3$ group
elements and $P + 4$ scalars. (TODO explain bytes)

TODO: Explain that the tables show the "high water" mark.

\begin{table}
	\centering
% BEGIN memory-3-5-3-128
% END memory-3-5-3-128
	\caption{Memory complexity for 3-of-5 FROST with 3 participants at the 128-bit security level. The cost is in bytes of memory allocated.}
\end{table}

\begin{table}
	\centering
% BEGIN memory-6-10-8-224
% END memory-6-10-8-224
	\caption{Memory complexity for 6-of-10 FROST with 8 participants at the 224-bit security level. The cost is in bytes of memory allocated.}
\end{table}

\begin{table}
	\centering
% BEGIN memory-600-1000-700-128
% END memory-600-1000-700-128
	\caption{Memory complexity for 600-of-1000 FROST with 700 participants at the 128-bit security level. The cost is in bytes of memory allocated.}
\end{table}

\subsection{Computational complexity}

The coordinator performs no significant computations in round 1. Participants
perform 2 fixed-base exponentiations, and (for the security levels presented in
this document) process 2 blocks of hash function input.

The significant computations occur during round 2.

Participants perform $2P$ group element multiplications, $P$ exponentiations,
$P + 1$ scalar additions, and $2P + 2$ scalar multiplications. The during
aggregation, the coordinator performs $2P$ group element multiplications, $P$
exponentiations, and $P$ scalar additions.

TODO: Fill out hash function complexity.

TODO: Convey the multi-exp by showing the number of terms in it, and then include
its muls etc. for a specific algorithm in the table (e.g. Pippenger).

\begin{table}
	\centering
% BEGIN computational-3-5-3-128
% END computational-3-5-3-128
	\caption{Computational complexity for 3-of-5 FROST with 3 participants at the 128-bit security level.}
\end{table}

\begin{table}
	\centering
% BEGIN computational-6-10-8-224
% END computational-6-10-8-224
	\caption{Computational complexity for 6-of-10 FROST with 8 participants at the 224-bit security level.}
\end{table}

\begin{table}
	\centering
% BEGIN computational-600-1000-700-128
% END computational-600-1000-700-128
	\caption{Computational complexity for 600-of-1000 FROST with 700 participants at the 128-bit security level.}
\end{table}
